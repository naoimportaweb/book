% Auth: Nicklas Vraa
% Docs: https://github.com/NicklasVraa/LiX
% Everything you need to know about this template is found in on the github repository above. Stars are very appreciated.

\documentclass{novel}

\lang      {english}
\title     {Manual Completo do Debian GNU/linux}
\subtitle  {Teoria e prática}
\authors   {Wellington Pinto de Oliveira}
\cover     {resources/capalivrov2.png}{resources/novel_back.pdf}
\license   {CC}{by-nc-sa}{3.0}
\isbn      {978-0201529838}
\publisher {AIED.com.br}
\edition   {1}{2023}



\note{
	Este conteúdo foi desenvolvido para auxiliar nas práticas acadêmicas em disciplinas de Sistemas Operacionais com GNU/Linux e Redes de Computadores, é um texto que reúne teoria e a prática, ou seja, respalda teoria com a prática.\\
	\newline Toda a prática pode ser validada automaticamente com o uso de um agente inteligente que auxilia na resolução dos erros implementados pelo aprendiz, sem a necessidade de intervenção de um instrutor neste primeiro momento, mas é claro, se nem com isso o aprendiz conseguir então é necessário a intervenção do educador.\\
	\newline O material também conta com vídeos explicativos, o objetivo dos vídeos é proporcionar estudos realizados por autodidatas ou aprendizes em tempo vago.\\
	\newline Todo o material está fundamentado em uma distribuição GNU/Linux Debian 12 de 64 bits por ser uma distribuição GNU/Linux conhecida e ser ótima para iniciantes. No primeiro capítulo o aprendiz terá acesso aos links e todo o ambiente é gratuito.\\
	\newline Procuro reforçar neste conteúdo todo o requisito para as provas de certificação Linux LPI e ainda iniciar o administrador Linux nas linguagens C/C++ e Python para resolver problemas por programação, e ainda uma boa explanação sobre Scripts Bash.\\
}

\blurb{Maecenas urna nisi, luctus nec lorem eu, vehicula varius eros. Nullam non quam tempus, ultrices lorem at, viverra felis. Nam eu ligula sodales, suscipit sem sed, bibendum augue.}

\begin{document}

\toc


Este conteúdo foi desenvolvido para auxiliar nas práticas acadêmicas em disciplinas de Sistemas Operacionais com GNU/Linux e Redes de Computadores, é um texto que reúne teoria e a prática, ou seja, respalda teoria com a prática.
Toda a prática pode ser validada automaticamente com o uso de um agente inteligente que auxilia na resolução dos erros implementados pelo aprendiz, sem a necessidade de intervenção de um instrutor neste primeiro momento, mas é claro, se nem com isso o aprendiz conseguir então é necessário a intervenção do educador.

O material também conta com vídeos explicativos, o objetivo dos vídeos é proporcionar estudos realizados por autodidatas ou aprendizes em tempo vago.

Todo o material está fundamentado em uma distribuição GNU/Linux Debian 12 de 64 bits por ser uma distribuição GNU/Linux conhecida e ser ótima para iniciantes. No primeiro capítulo o aprendiz terá acesso aos links e todo o ambiente é gratuito.

Procuro reforçar neste conteúdo todo o requisito para as provas de certificação Linux LPI e ainda iniciar o administrador Linux nas linguagens C/C++ e Python para resolver problemas por programação, e ainda uma boa explanação sobre Scripts Bash. 


Este conteúdo foi desenvolvido para auxiliar nas práticas acadêmicas em disciplinas de Sistemas Operacionais com GNU/Linux e Redes de Computadores, é um texto que reúne teoria e a prática, ou seja, respalda teoria com a prática.
Toda a prática pode ser validada automaticamente com o uso de um agente inteligente que auxilia na resolução dos erros implementados pelo aprendiz, sem a necessidade de intervenção de um instrutor neste primeiro momento, mas é claro, se nem com isso o aprendiz conseguir então é necessário a intervenção do educador.

O material também conta com vídeos explicativos, o objetivo dos vídeos é proporcionar estudos realizados por autodidatas ou aprendizes em tempo vago.

Todo o material está fundamentado em uma distribuição GNU/Linux Debian 12 de 64 bits por ser uma distribuição GNU/Linux conhecida e ser ótima para iniciantes. No primeiro capítulo o aprendiz terá acesso aos links e todo o ambiente é gratuito.

Procuro reforçar neste conteúdo todo o requisito para as provas de certificação Linux LPI e ainda iniciar o administrador Linux nas linguagens C/C++ e Python para resolver problemas por programação, e ainda uma boa explanação sobre Scripts Bash. 


Este conteúdo foi desenvolvido para auxiliar nas práticas acadêmicas em disciplinas de Sistemas Operacionais com GNU/Linux e Redes de Computadores, é um texto que reúne teoria e a prática, ou seja, respalda teoria com a prática.
Toda a prática pode ser validada automaticamente com o uso de um agente inteligente que auxilia na resolução dos erros implementados pelo aprendiz, sem a necessidade de intervenção de um instrutor neste primeiro momento, mas é claro, se nem com isso o aprendiz conseguir então é necessário a intervenção do educador.

O material também conta com vídeos explicativos, o objetivo dos vídeos é proporcionar estudos realizados por autodidatas ou aprendizes em tempo vago.

Todo o material está fundamentado em uma distribuição GNU/Linux Debian 12 de 64 bits por ser uma distribuição GNU/Linux conhecida e ser ótima para iniciantes. No primeiro capítulo o aprendiz terá acesso aos links e todo o ambiente é gratuito.

Procuro reforçar neste conteúdo todo o requisito para as provas de certificação Linux LPI e ainda iniciar o administrador Linux nas linguagens C/C++ e Python para resolver problemas por programação, e ainda uma boa explanação sobre Scripts Bash. 


Este conteúdo foi desenvolvido para auxiliar nas práticas acadêmicas em disciplinas de Sistemas Operacionais com GNU/Linux e Redes de Computadores, é um texto que reúne teoria e a prática, ou seja, respalda teoria com a prática.
Toda a prática pode ser validada automaticamente com o uso de um agente inteligente que auxilia na resolução dos erros implementados pelo aprendiz, sem a necessidade de intervenção de um instrutor neste primeiro momento, mas é claro, se nem com isso o aprendiz conseguir então é necessário a intervenção do educador.

O material também conta com vídeos explicativos, o objetivo dos vídeos é proporcionar estudos realizados por autodidatas ou aprendizes em tempo vago.

Todo o material está fundamentado em uma distribuição GNU/Linux Debian 12 de 64 bits por ser uma distribuição GNU/Linux conhecida e ser ótima para iniciantes. No primeiro capítulo o aprendiz terá acesso aos links e todo o ambiente é gratuito.

Procuro reforçar neste conteúdo todo o requisito para as provas de certificação Linux LPI e ainda iniciar o administrador Linux nas linguagens C/C++ e Python para resolver problemas por programação, e ainda uma boa explanação sobre Scripts Bash. 



\end{document}
